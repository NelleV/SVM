\documentclass{article}
%\usepackage[latin1]{inputenc}
\usepackage{graphicx,amssymb,amsmath,amsbsy,MnSymbol} % extensions pour maths avancées
\usepackage{graphicx,mathenv}           % extensions pour figures
\usepackage[T1]{fontenc}        % pour les charactères accentués 
\usepackage[utf8]{inputenc} 
\usepackage{multicol}
\usepackage{wrapfig}
\usepackage{stmaryrd} % Pour les crochets d'ensemble d'entier
\usepackage{float}  % Pour placer les images là ou JE veux.

\DeclareMathOperator{\tr}{tr}
\DeclareMathOperator{\argmax}{argmax}
\DeclareMathOperator{\cov}{cov}


\setlength{\parindent}{0.0in}
\setlength{\parskip}{0.1in}
\setlength{\topmargin}{-0.4in}
\setlength{\topskip}{0.7in}    % between header and text
\setlength{\textheight}{9in} % height of main text
\setlength{\textwidth}{6in}    % width of text
\setlength{\oddsidemargin}{0in} % odd page left margin
\setlength{\evensidemargin}{0in} % even page left margin
%
%% Quelques raccourcis clavier :
\def\slantfrac#1#2{\kern.1em^{#1}\kern-.3em/\kern-.1em_{#2}}
\def\b#1{\mathbf{#1}}
\def\bs#1{\boldsymbol{#1}}
\def\m#1{\mathrm{#1}}
\bibliographystyle{acm}
%
\newcommand{\greeksym}[1]{{\usefont{U}{psy}{m}{n}#1}}
\newcommand{\inc}{\mbox{\small\greeksym{d}\hskip 0.05ex}}%
\pagenumbering{arabic}
\date{\today}
\title{Méthodes à noyaux - DM2}
\author{Nelle Varoquaux}
\begin{document}
\maketitle

\section{Exercice 1}
Soit $X = \(x_1, x_2, \dots, x_n\) \in \mathbb{R}^n$ et 
$Y = \(y_1, y_2, \dots, y_n\) \in \mathbb{R}^n$ . On définit:
\begin{equation*}
\cov_n (X, Y) = \mathbf{E}_n(XY) - \mathbf{E}_n(X)\mathbf{E}_n(Y)
\end{equation*}

avec $\mathbf{E}_n(U) = \(\sum_{i = 1}^n u_i / n\)$. Considérons par ailleurs
le critère suivant:

\begin{equation*}
C_n^K(X, Y ) = \max_{f, g \in \mathcal{B}_K} \cov_n(f(X), g(Y))
\end{equation*}

où $K$ est un noyau défini positif sur $\mathbb{R}$, $\mathcal{B_K}$ la
boule unité du RKHS de $K$, et $f(U) = (f(u_1), f(u_2), \dots, f(u_n))$

\subsection{Question 1}
Considérons le noyau linéaire : $K(a, b) = ab$.

\begin{align*}
C_n^K(X, Y ) & = & \max_{f, g \in \mathcal{B}_K} \cov_n(f(X), g(Y)) \\
	     & = & \max_{f, g \in \mathcal{B}_K} \mathbf{E}_n(f(X) g(Y)) - \mathbf{E}_n(f(X))\mathbf{E}_n(g(Y)) \\
	     & = & \max_{f, g \in \mathcal{B}_K} \frac{\sum_{i = 1}^n \langle f, K_{x_i}\rangle \langle g, K_{y_i}\rangle}{n^2} -
						 \frac{\sum_{i = 1}^n \langle f, K_{x_i}\rangle \sum_{i = 1}^n \langle g, K_{y_i}\rangle}{n^2} 
\end{align*}

On peut poser $f(x) = wx$ et $g(wx) = vx$. Or $f, g \in \mathcal{B}_K$. Donc
$w = v = 1$.
On a donc:

\begin{align*}
C_n^K(X, Y ) & = & \frac{\sum_{i = 1}^n x_i y_i}{n^2} - \frac{\sum_{i = 1}^n x_i \sum_{i = 1}^n y_i}{n^2}
\end{align*}


\section{Exercice 2}
\subsection{Question 1}
\end{document}
