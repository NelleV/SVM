\documentclass{article}
%\usepackage[latin1]{inputenc}
\usepackage{graphicx,amssymb,amsmath,amsbsy,MnSymbol} % extensions pour maths avancées
\usepackage{graphicx,mathenv}           % extensions pour figures
\usepackage[T1]{fontenc}        % pour les charactères accentués 
\usepackage[utf8]{inputenc} 
\usepackage{multicol}
\usepackage{wrapfig}
\usepackage{stmaryrd} % Pour les crochets d'ensemble d'entier
\usepackage{float}  % Pour placer les images là ou JE veux.

\DeclareMathOperator{\tr}{tr}
\DeclareMathOperator{\argmax}{argmax}
\DeclareMathOperator{\sinc}{sinc}


\setlength{\parindent}{0.0in}
\setlength{\parskip}{0.1in}
\setlength{\topmargin}{-0.4in}
\setlength{\topskip}{0.7in}    % between header and text
\setlength{\textheight}{9in} % height of main text
\setlength{\textwidth}{6in}    % width of text
\setlength{\oddsidemargin}{0in} % odd page left margin
\setlength{\evensidemargin}{0in} % even page left margin
%
%% Quelques raccourcis clavier :
\def\slantfrac#1#2{\kern.1em^{#1}\kern-.3em/\kern-.1em_{#2}}
\def\b#1{\mathbf{#1}}
\def\bs#1{\boldsymbol{#1}}
\def\m#1{\mathrm{#1}}
\bibliographystyle{acm}
%
\newcommand{\greeksym}[1]{{\usefont{U}{psy}{m}{n}#1}}
\newcommand{\inc}{\mbox{\small\greeksym{d}\hskip 0.05ex}}%
\pagenumbering{arabic}
\date{\today}
\title{Méthodes à noyaux - DM4}
\author{Nelle Varoquaux}
\begin{document}
\maketitle

\section{Exercice 1}
Soit

\begin{equation*}
I(x) = \begin{cases}
       1 & \mbox{si} -1 \leq x \leq 1 \\
       0 & \mbox{sinon}
       \end{cases}
\end{equation*}

Posons $B_n = I^{*n}$ pour $n \in \mathbb{N}_*$.

Etudions la transformée de Fourir inverse de $B_n$:
\begin{align*}
\breve{B}_n(x) & = & \frac{1}{2 \pi} \int_{- \infty}^{\infty} e^{ix\omega}
	     \int_{- \infty}^{\infty} I(u) B_{n- 1}(\omega - u) du d\omega\\
	   & = & \frac{1}{2 \pi} \int_{- \infty}^{\infty} e^{ix\omega}
		 \int_{- 1}^{1} B_{n- 1}(\omega - u) du d \omega\\
\end{align*}

Par Fubini, on a donc:
\begin{align*}
\breve{B}_n(x) & = & \frac{1}{2 \pi} \int_{- 1}^{1} \int_{- \infty}^{\infty}
	           e^{ix\omega} B_{n - 1}(\omega - u) d\omega du \\
	       & = & \frac{1}{2 \pi} \int_{- 1}^{1} \int_{- \infty}^{\infty}
	           e^{ixu} e^{-ixt }B_{n - 1}(t) dt du \\
	       & = & \frac{1}{2 \pi} \int_{- 1}^{1} e^{ixu}  \int_{- \infty}^{\infty}
	           e^{-ixt }B_{n - 1}(t) dt du \\
	       & = & \frac{1}{2 \pi} \int_{- 1}^{1} e^{ixu} \breve{B}_{n - 1}(t) dt du \\
	       & = & \frac{1}{2 \pi} \sinc(x) \breve{B}_{n - 1}(x) \\
\end{align*}

On a donc par récurrence:

\begin{align*}
\breve{B}_n(x) & = & \frac{1}{(2 \pi)^n} \sinc(x)^n \\
\end{align*}

\section{Exercice 2}


\begin{itemize}

\item Posons $\forall x, y \in \mathbb{R}$, $K_1(x, y) = \frac{1}{2 - e^{- \| x - y\|^2}}$

On a donc:

\begin{align*}
K_1(x, y) & = & \frac{1}{2 - e^{- \| x - y\|^2}} \\
	  & = & \frac{1}{2} \frac{1}{1 - \frac{e^{- \| x - y\|^2}}{2}} \\
	  & = & \frac{1}{2} \sum_{n = 1}^\infty \(\frac{e^{- \| x - y\|^2}}{2}\)^n
\end{align*}

Étudions maintenant chaque terme de la suite $K^{(n)}(x, y) = \(\frac{e^{- \| x -
y\|^2}}{2}\)^n $. On a alors $\kappa^{(n)}(\omega) = \frac{e^{-
n\|\omega\|^2)}}{2^n}$. Donc par le théorème de Bochner, $\forall n, K^{(n)}$
est défini positif. Chaque terme de la série de $K_1$ est donc définie
positif. On peut donc en déduire que $K_1$ est défini positif.

\item Posons $\forall x, y \in \mathbb{R}$, $K_2(x, y) = \max\(0, 1 - |x - y|\)$

Étudions la transformée de Fourier inverse de: $K_2(u) = \max\(0, 1 - |u|\)$

\begin{align*}
\breve{K}_2(u) & = & \frac{1}{2\pi} \int_{- \infty}^{\infty} e^{iu\omega}
		     \max\(0, 1 - |\omega|\) d\omega \\
	       & = & \frac{1}{2\pi} \int_{- 1}^{1} e^{iu\omega} \(1 - |\omega|\) d\omega \\
	       & = & \frac{1}{2 \pi} \[\sinc(x) - 2
		     \int_{0}^1e^{iu\omega}\omega d\omega\] \\
	       & = & \frac{1}{2 \pi} \sinc(x)
\end{align*}

Donc $\breve{K}_2(u)$ prends des valeurs négatives: par le théorème de
Bochner, $K_2(x, y)$ n'est donc pas défini positif.

\item Posons $\forall x, y \in \mathbb{R}$, $K_3(x, y) =\frac{1}{1 + x + y}$

Commençons par démontrer que les semi-charactères bornées sur le semi groupe
abélien $(\mathbb{R}, +, Id)$ sont exactement l'ensemble des fonctions
$s \in \mathbb{R}_+ \rightarrow \rho_a(s) = e^{-as}$.
\begin{itemize}
\item $\rho(x)$ est continue en 0
\item Puisque $\rho(x + y) = \rho(x)\rho(y)$, $\rho$ est $\mathcal{C}^1$
\item Posons $\varrho$ la primitive de $\rho$. On a donc $\varrho(x + y) =
\int_{0}^{x + y} \rho(u) du$. Donc:
\begin{align*}
\varrho(x + y) & = & \int_{0}^{x + y}\rho(u) du\\
	       & = & \int_{0}^x \rho(u) du + \int_{x}^{x + y} \rho(u) du \\
	       & = & \int_{0}^x \rho(u) du + \int_{x}^{x + y} \rho(u) du \\
	       & = & \int_{0}^x \rho(u) du + \int_{0}^{y} \rho(u + x) du \\
	       & = & \int_{0}^x \rho(u) du + \int_{0}^{y} \rho(x)\rho(u) du \\
\end{align*}
% FIXME
Donc $\rho$ est $\mathcal{C}^2$
\end{itemize}
Posons $d \mu(a) = e^{-a} da$. On a alors:

\begin{align*}
\int_0^\infty e^{-as}d\mu(a) & = & \int_0^\infty e^{-as}a^{-a}da \\
			     & = & \int_0^\infty e^{-a(1 + s)}da \\
			     & = &\frac{1}{1 + s}
\end{align*}

Donc $K_3$ est un noyau semi définie positif.

\end{itemize}


\section{Exercice 3}

Montrons que $\max(x + y - 1, 0)$ est une loi associative interne dans $[0,
1]$.

\begin{align*}
\max \( \max(x + y - 1, 0) + z - 1, 0 \) & = & \max \( \max(x + y + z - 2, z -
1), 0\)
\end{align*}

Or $z - 1 \leq 0$, puisque $z \in [0, 1]$. Donc :

\begin{align*}
\max \( \max(x + y - 1, 0) + z - 1, 0 \) & = & \max \( \max(x + y + z - 2, 0\)
\end{align*}

Donc $\max(x + y - 1, 0)$ est une loi associative. De plus, $\forall (x, y)
\in \mathbb{R}^2, x + y - 1 \in \mathbb{R}$, donc elle est interne.

On peut donc définir un semi-groupe: $([0, 1], \circ)$, avec $x \circ y =
\max(x + y - 1, 0)$, d'élément neutre $1$. $([0, 1], \circ, Id)$ est alors un
semi-groupe avec involution.

On cherche les fonctions $\phi$ telles que $\phi(\max(x + y - 1, 0))$ soit
défini positif. On a donc:

\begin{align*}
a_1^2 \phi(\max(2x - 1, 0)) + 2 a_1 a_2 \phi(\max(x + y - 1, 0)) + a_2^2
\phi(\max(2y - 1, 0)) \geq 0
\end{align*}

Supposons maintenant $x = 0$ et $y$ tel que $2y - 1 > 0$. Posons $u = 2y - 1$.
On a alors:
\begin{align*}
a_1^2 \phi(0) + 2 a_1 a_2 \phi(0) + a_2^2 \phi(u) & \geq  & 0 \\
\end{align*}

Étudions le discriminant: $\Delta = 4 a_2^2 \phi(0)^2 - 4 \phi(0) a_2^2
\phi(u)$. On sait que $\Delta \leq 0$. Donc $\phi(0)^2 - \phi(0)\phi(u) \leq
0$. On a donc $\phi(0) \leq \phi(u)$, et ce $\forall u$. Donc $\min (\phi) =
\phi(0)$.

Prenons maintenant $x$ quelconque et $y = 1$. On a alors:
\begin{align*}
a_1^2 \phi(\max(2x - 1, 0)) + 2 a_1 a_2 \phi(x) + a_2^2 \phi(1) & \geq  & 0 \\
\end{align*}

Le discriminant est donc $\Delta = 4 a_1^2 \phi(x)^2 - 4 a_1^2 \phi(\max(2x -
1, 0)) \phi(1)$. Puisque $\Delta \leq 0$, on a donc $\phi(x)^2 \leq
\phi(\max(2x - 1, 0)) \phi(1)$


Se présentent alors à nous deux cas:

\begin{itemize}
\item $x \leq \frac{1}{2}$: on a alors $\max(2x - 1, 0) = 0$
et donc $\phi(x)^2 \leq \phi(0) \phi(1)$; or $\phi(0) \leq \phi(x)$, donc
$\phi(x)^2 \leq \phi(x)\phi(1)$. Puisque $\phi(x) \geq 0$ ($\phi$ étant
définie positive), $\phi(x) \leq \phi(1)$

\item $1 > x > \frac{1}{2}$; on a alors $\max(2x - 1, 0) = 2x - 1$, et donc
$\phi(x)^2 \leq \phi(2x - 1) \phi(1)$.

Définissons la suite $(u_n)_{n \in \mathbb{N}}$ tel que $u_0 = x$ et $u_{n +
1} = 2u_n - 1$. Cette suite est strictement décroissante, et tends vers
$-\infty$. Donc, il existe $n$ tel que $u_n \geq \frac{1}{2}$, et donc tel que
$\phi(u_n) < \phi(1)$. Or $\phi(u_{n})^2 < \phi(u_{n - 1})\phi(1)$, donc
$\phi(u_{n - 1}) < 1$. Par récurrence on obtient donc que $\phi(u_0) = \phi(x) <
\phi(1)$, et ce $\forall x \in ]\frac{1}{2}, 1[$.
\end{itemize}

Donc $\phi(1) = \max \phi(x)$. $\phi$ est donc bornée.

On peut donc en déduire que les fonctions $\phi$ telles que $\phi(max(x + y -
1)$ sont définies positives sont aussi bornées. On peut donc les écrire sous
la forme:

\begin{align*}
\phi(s) & = & \int_0^\infty \rho_a(s)d\mu(a) + b \rho_\infty
\end{align*}

$\rho_a$ étant les semi-charactères bornés sur $([0, 1], \circ)$,

\section{Exercice 4}

Considérons le semi-groupe $(\mathbb{R}, \max, Id)$

\end{document}
