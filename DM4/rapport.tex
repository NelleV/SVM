\documentclass{article}
%\usepackage[latin1]{inputenc}
\usepackage{graphicx,amssymb,amsmath,amsbsy,MnSymbol} % extensions pour maths avancées
\usepackage{graphicx,mathenv}           % extensions pour figures
\usepackage[T1]{fontenc}        % pour les charactères accentués 
\usepackage[utf8]{inputenc} 
\usepackage{multicol}
\usepackage{wrapfig}
\usepackage{stmaryrd} % Pour les crochets d'ensemble d'entier
\usepackage{float}  % Pour placer les images là ou JE veux.

\DeclareMathOperator{\tr}{tr}
\DeclareMathOperator{\argmax}{argmax}
\DeclareMathOperator{\sinc}{sinc}


\setlength{\parindent}{0.0in}
\setlength{\parskip}{0.1in}
\setlength{\topmargin}{-0.4in}
\setlength{\topskip}{0.7in}    % between header and text
\setlength{\textheight}{9in} % height of main text
\setlength{\textwidth}{6in}    % width of text
\setlength{\oddsidemargin}{0in} % odd page left margin
\setlength{\evensidemargin}{0in} % even page left margin
%
%% Quelques raccourcis clavier :
\def\slantfrac#1#2{\kern.1em^{#1}\kern-.3em/\kern-.1em_{#2}}
\def\b#1{\mathbf{#1}}
\def\bs#1{\boldsymbol{#1}}
\def\m#1{\mathrm{#1}}
\bibliographystyle{acm}
%
\newcommand{\greeksym}[1]{{\usefont{U}{psy}{m}{n}#1}}
\newcommand{\inc}{\mbox{\small\greeksym{d}\hskip 0.05ex}}%
\pagenumbering{arabic}
\date{\today}
\title{Méthodes à noyaux - DM4}
\author{Nelle Varoquaux}
\begin{document}
\maketitle

\section{Exercice 1}
Soit

\begin{equation*}
I(x) = \begin{cases}
       1 & \mbox{si} -1 \leq x \leq 1 \\
       0 & \mbox{sinon}
       \end{cases}
\end{equation*}

Posons $B_n = I^{*n}$ pour $n \in \mathbb{N}_*$.

Etudions la transformée de Fourir inverse de $B_n$:
\begin{align*}
\breve{B}_n(x) & = & \frac{1}{2 \pi} \int_{- \infty}^{\infty} e^{ix\omega}
	     \int_{- \infty}^{\infty} I(u) B_{n- 1}(\omega - u) du d\omega\\
	   & = & \frac{1}{2 \pi} \int_{- \infty}^{\infty} e^{ix\omega}
		 \int_{- 1}^{1} B_{n- 1}(\omega - u) du d \omega\\
\end{align*}

Par Fubini, on a donc:
\begin{align*}
\breve{B}_n(x) & = & \frac{1}{2 \pi} \int_{- 1}^{1} \int_{- \infty}^{\infty}
	           e^{ix\omega} B_{n - 1}(\omega - u) d\omega du \\
	       & = & \frac{1}{2 \pi} \int_{- 1}^{1} \int_{- \infty}^{\infty}
	           e^{ixu} e^{-ixt }B_{n - 1}(t) dt du \\
	       & = & \frac{1}{2 \pi} \int_{- 1}^{1} e^{ixu}  \int_{- \infty}^{\infty}
	           e^{-ixt }B_{n - 1}(t) dt du \\
	       & = & \frac{1}{2 \pi} \int_{- 1}^{1} e^{ixu} \breve{B}_{n - 1}(t) dt du \\
	       & = & \frac{1}{2 \pi} \sinc(x) \breve{B}_{n - 1}(x) \\
\end{align*}

On a donc par récurrence:

\begin{align*}
\breve{B}_n(x) & = & \frac{1}{(2 \pi)^n} \sinc(x)^n \\
\end{align*}

\section{Exercice 2}


\begin{itemize}

\item Posons $\forall x, y \in \mathbb{R}$, $K_1(x, y) = \frac{1}{2 - e^{- \| x - y\|^2}}$

On a donc:

\begin{align*}
K_1(x, y) & = & \frac{1}{2 - e^{- \| x - y\|^2}} \\
	  & = & \frac{1}{2} \frac{1}{1 - \frac{e^{- \| x - y\|^2}}{2}} \\
	  & = & \frac{1}{2} \sum_{n = 1}^\infty \(\frac{e^{- \| x - y\|^2}}{2}\)^n
\end{align*}

Étudions maintenant chaque terme de la suite $K^{(n)}(x, y) = \(\frac{e^{- \| x -
y\|^2}}{2}\)^n $. On a alors $\kappa^{(n)}(\omega) = \frac{e^{-
n\|\omega\|^2)}}{2^n}$. Donc par le théorème de Bochner, $\forall n, K^{(n)}$
est défini positif. Chaque terme de la série de $K_1$ est donc définie
positif. On peut donc en déduire que $K_1$ est défini positif.

\item Posons $\forall x, y \in \mathbb{R}$, $K_2(x, y) = \max\(0, 1 - |x - y|\)$

Étudions la transformée de Fourier inverse de: $K_2(u) = \max\(0, 1 - |u|\)$

\begin{align*}
\breve{K}_2(u) & = & \frac{1}{2\pi} \int_{- \infty}^{\infty} e^{iu\omega}
		     \max\(0, 1 - |\omega|\) d\omega \\
	       & = & \frac{1}{2\pi} \int_{- 1}^{1} e^{iu\omega} \(1 - |\omega|\) d\omega \\
	       & = & \frac{1}{2 \pi} \[\sinc(x) - 2
		     \int_{0}^1e^{iu\omega}\omega d\omega\] \\
	       & = & \frac{1}{2 \pi} \sinc(x)
\end{align*}

Donc $\breve{K}_2(u)$ prends des valeurs négatives: par le théorème de
Bochner, $K_2(x, y)$ n'est donc pas défini positif.

\item Posons $\forall x, y \in \mathbb{R}$, $K_3(x, y) =\frac{1}{1 + x + y}$


\end{itemize}

\end{document}
